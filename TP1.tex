%%%%%%%%%%%%%%%%%%%%%%%%%%%%%%%%%%%%%%%%%%%%%%%%%%%%%%%%%%%%%%%%%%%%%%%%%%%%%%%
% Definici�n del tipo de documento.                                           %
% Posibles tipos de papel: a4paper, letterpaper, legalpapper                  %
% Posibles tama�os de letra: 10pt, 11pt, 12pt                                 %
% Posibles clases de documentos: article, report, book, slides                %
%%%%%%%%%%%%%%%%%%%%%%%%%%%%%%%%%%%%%%%%%%%%%%%%%%%%%%%%%%%%%%%%%%%%%%%%%%%%%%%
\documentclass[a4paper,10pt]{article}


%%%%%%%%%%%%%%%%%%%%%%%%%%%%%%%%%%%%%%%%%%%%%%%%%%%%%%%%%%%%%%%%%%%%%%%%%%%%%%%
% Los paquetes permiten ampliar las capacidades de LaTeX.                     %
%%%%%%%%%%%%%%%%%%%%%%%%%%%%%%%%%%%%%%%%%%%%%%%%%%%%%%%%%%%%%%%%%%%%%%%%%%%%%%%

% Paquete para inclusi�n de gr�ficos.
\usepackage{graphicx}
%Packete para codigo
\usepackage{listings}


% Paquete para definir la codificaci�n del conjunto de caracteres usado
% (latin1 es ISO 8859-1).
\usepackage[latin1]{inputenc}

% Paquete para definir el idioma usado.
\usepackage[spanish]{babel}


% T�tulo principal del documento.
\title{		\textbf{Informe TP1}}

% Informaci�n sobre los autores.
\author{Ignacio Manes, \textit{Padr�n Nro. 97.092}                     \\
            \texttt{ nacho.maness@gmail.com }                                              \\
            \texttt{ Pablo Israel} \textit{Padr�n Nro. 95.849}                     \\
            \texttt{ direcci�n de e-mail }											    \\[2.5ex]
            \normalsize{Grupo Nro. 0 - 2do. Cuatrimestre de 2006}                       \\
            \normalsize{66.20 Organizaci�n de Computadoras}                             \\
            \normalsize{Facultad de Ingenier�a, Universidad de Buenos Aires}            \\
       }
\date{}



\begin{document}

% Inserta el t�tulo.
\maketitle

% Quita el n�mero en la primer p�gina.
\thispagestyle{empty}

\newpage

\section{Introducci�n}

Este informe tratar� sobre el trabajo pr�ctico n�1 de la asignatura Organizaci�n de Computadoras,en el cu�l desarrollaremos el juego de la vida de Conway. Para esta primer entrega, se ha desarrollado solo en c, por cuestiones de dificultad en el lenguaje, que nos impidieron avanzar con la secci�n de assembler.


\section{Documentaci�n relevante al dise�o e implementaci�n}

Entre la documentaci�n de relevancia podemos ver el makefile.
En cuanto al dise�o, se encapsulo en funciones para mayor legibilidad.
Los problemas que se nos presentaron fueron el uso del pbm y el malloc, ya que ninguno de nosotros habiamos usado c antes, y mas que todo en el malloc tuvimos problemas con la aritmetica en el array de punteros.
Se definio para el pbm un ancho y un alto de 50 pixeles por cuadrado de cada celda de la matriz.


\section{Ejecuciones}
	\subsection{Terminal}
\begin{lstlisting}
./conway 10 20 20 glider
Leyendo estado inicial...
Grabando glider_0.pbm
Grabando glider_1.pbm
Grabando glider_2.pbm
Grabando glider_3.pbm
Grabando glider_4.pbm
Grabando glider_5.pbm
Grabando glider_6.pbm
Grabando glider_7.pbm
Grabando glider_8.pbm
Grabando glider_9.pbm
Grabando glider_10.pbm
Listo

./conway 10 20 20 pento
Leyendo estado inicial...
Grabando pento_0.pbm
Grabando pento_1.pbm
Grabando pento_2.pbm
Grabando pento_3.pbm
Grabando pento_4.pbm
Grabando pento_5.pbm
Grabando pento_6.pbm
Grabando pento_7.pbm
Grabando pento_8.pbm
Grabando pento_9.pbm
Grabando pento_10.pbm
Listo

./conway 10 20 20 sapo
Leyendo estado inicial...
Grabando sapo_0.pbm
Grabando sapo_1.pbm
Grabando sapo_2.pbm
Grabando sapo_3.pbm
Grabando sapo_4.pbm
Grabando sapo_5.pbm
Grabando sapo_6.pbm
Grabando sapo_7.pbm
Grabando sapo_8.pbm
Grabando sapo_9.pbm
Grabando sapo_10.pbm
Listo
\end{lstlisting}
Dichos pmb se encuentran adjuntados junto con el c�digo fuente.
\end{document}
